\documentclass[12pt]{article}
\usepackage{amsmath,amsthm,amssymb}

\textheight=22cm
\textwidth=15cm
\oddsidemargin=5mm
\evensidemargin=5mm

\begin{document}

\begin{center}

%%%%%%%%%%%%%%%%%%%%%%%%%%%%%%%%%%%%%%%%%%%%%%%%%%%%%%%%%%%%%%%%%%%%%%
% TITLE - please fill in the title below:
%%%%%%%%%%%%%%%%%%%%%%%%%%%%%%%%%%%%%%%%%%%%%%%%%%%%%%%%%%%%%%%%%%%%%%
{\bf\Large
Quantifying Movement Variability with \\ 
Nonlinear Dynamics for 
Human-Humanoid Interaction 
}

\vspace{8mm}

%%%%%%%%%%%%%%%%%%%%%%%%%%%%%%%%%%%%%%%%%%%%%%%%%%%%%%%%%%%%%%%%%%%%%%
% First author's name and affiliation:
%%%%%%%%%%%%%%%%%%%%%%%%%%%%%%%%%%%%%%%%%%%%%%%%%%%%%%%%%%%%%%%%%%%%%%

{\sc\large Miguel Xochicale}

\vspace{3mm}

{\em School of Biomedical Engineering and Imaging Sciences \\
King's College London, UK \\
E-mail: miguel.xochicale@kcl.ac.uk}

\vspace{5mm}




%%%%%%%%%%%%%%%%%%%%%%%%%%%%%%%%%%%%%%%%%%%%%%%%%%%%%%%%%%%%%%%%%%%%%
% Please repeat for any additional authors
%%%%%%%%%%%%%%%%%%%%%%%%%%%%%%%%%%%%%%%%%%%%%%%%%%%%%%%%%%%%%%%%%%%%%
\end{center}
\vspace{8mm}
% Please delete as appropriate
\noindent {\bf Presentation type:}  Standard Talk.
%
\vspace{8mm}

%%%%%%%%%%%%%%%%%%%%%%%%%%%%%%%%%%%%%%%%%%%%%%%%%%%%%%%%%%%%%%%%%%%%%%
% ABSTRACT: Please insert your abstract below:
%%%%%%%%%%%%%%%%%%%%%%%%%%%%%%%%%%%%%%%%%%%%%%%%%%%%%%%%%%%%%%%%%%%%%%
Movement variability occurs in motor performance 
of multiple repetitions of a task in a certain environment.
The challenge with quantifying movement variability 
is that traditional methods in time domain and frequency domain 
fail to detect tiny modulations in frequency or phase of time series. 
In this talk, methods of nonlinear dynamics (i.e., 
reconstructed state space, uniform time-delay embedding, 
recurrence plots and recurrence quantification analysis) 
are introduced with the aim of quantifying movement variability
in human-humanoid interaction.
Additionally, experiments are presented where 
twenty right-handed healthy participants 
imitated vertical and horizontal arm movements in normal 
and faster speed from an humanoid robot.
From time series data collected with inertial measurements
sensors, four window lengths and three levels of smoothed 
time series were analysed with nonlinear dynamics methods.
Then weaknesses and robustness of nonlinear dynamics 
methods for raw data and post-processed data are presented
and therefore concluded that 
Shannon entropy values from Recurrence Quantification Analysis 
were well distributed and showed variation in all the 
conditions of time series data.
With that in mind, it is going to be highlighted 
the potential of nonlinear dynamics methods to enhance the 
development of better diagnostic tools 
for various applications in medical robotics or 
for new forms of human-humanoid interaction.
\vspace{5mm}

%%%%%%%%%%%%%%%%%%%%%%%%%%%%%%%%%%%%%%%%%%%%%%%%%%%%%%%%%%%%%%%%%%%%%%%%%%%%%%%%%%%%%%%%%%%%%%%%%%%%%%%%%%
% BIBLIOGRAPHY: Please insert the references in alphabetical order, after the surname of the first author.
%%%%%%%%%%%%%%%%%%%%%%%%%%%%%%%%%%%%%%%%%%%%%%%%%%%%%%%%%%%%%%%%%%%%%%%%%%%%%%%%%%%%%%%%%%%%%%%%%%%%%%%%%%
\begin{thebibliography}{9}

\bibitem{Reference 1} 
Xochicale M. and Baber C. (2018)
\newblock Strengths and Weaknesses of Recurrent Quantification Analysis in the context of Human-Humanoid Interaction.
\newblock {\em preprint, arXiv} https://arxiv.org/abs/1810.09249.

\bibitem{Reference 2} 
{Xochicale}, M. (2019).
\newblock Nonlinear Analysis to Quantify Movement Variability in Human-Humanoid Interaction.
\newblock {\em {PhD} dissertation}, University of Birmingham, United Kingdom.
\newblock https://github.com/mxochicale/phd-thesis


\bibitem{Reference 3} 
Stergiou N. and  Leslie M. D. (2011)
\newblock Human movement variability, nonlinear dynamics, and pathology: Is there a connection?.
\newblock {\em Human Movement Science}, 30(5):869 -- 888.


\bibitem{Reference 4} 
Marwan, N., Romano, M.~C., Thiel, M., and Kurths, J. (2007).
\newblock Recurrence plots for the analysis of complex systems.
\newblock {\em Physics Reports}, 438(5):237 -- 329.

\bibitem{Reference 5}  
Davids, K., Glazier, P., Araujo, D., and Bartlett, R. (2003).
\newblock Movement systems as dynamical systems.
\newblock {\em Sports Medicine}, 33(4):245--260.



\end{thebibliography}

\end{document}
